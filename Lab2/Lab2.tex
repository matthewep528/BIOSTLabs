% Options for packages loaded elsewhere
\PassOptionsToPackage{unicode}{hyperref}
\PassOptionsToPackage{hyphens}{url}
%
\documentclass[
]{article}
\usepackage{amsmath,amssymb}
\usepackage{iftex}
\ifPDFTeX
  \usepackage[T1]{fontenc}
  \usepackage[utf8]{inputenc}
  \usepackage{textcomp} % provide euro and other symbols
\else % if luatex or xetex
  \usepackage{unicode-math} % this also loads fontspec
  \defaultfontfeatures{Scale=MatchLowercase}
  \defaultfontfeatures[\rmfamily]{Ligatures=TeX,Scale=1}
\fi
\usepackage{lmodern}
\ifPDFTeX\else
  % xetex/luatex font selection
\fi
% Use upquote if available, for straight quotes in verbatim environments
\IfFileExists{upquote.sty}{\usepackage{upquote}}{}
\IfFileExists{microtype.sty}{% use microtype if available
  \usepackage[]{microtype}
  \UseMicrotypeSet[protrusion]{basicmath} % disable protrusion for tt fonts
}{}
\makeatletter
\@ifundefined{KOMAClassName}{% if non-KOMA class
  \IfFileExists{parskip.sty}{%
    \usepackage{parskip}
  }{% else
    \setlength{\parindent}{0pt}
    \setlength{\parskip}{6pt plus 2pt minus 1pt}}
}{% if KOMA class
  \KOMAoptions{parskip=half}}
\makeatother
\usepackage{xcolor}
\usepackage[margin=1in]{geometry}
\usepackage{color}
\usepackage{fancyvrb}
\newcommand{\VerbBar}{|}
\newcommand{\VERB}{\Verb[commandchars=\\\{\}]}
\DefineVerbatimEnvironment{Highlighting}{Verbatim}{commandchars=\\\{\}}
% Add ',fontsize=\small' for more characters per line
\usepackage{framed}
\definecolor{shadecolor}{RGB}{248,248,248}
\newenvironment{Shaded}{\begin{snugshade}}{\end{snugshade}}
\newcommand{\AlertTok}[1]{\textcolor[rgb]{0.94,0.16,0.16}{#1}}
\newcommand{\AnnotationTok}[1]{\textcolor[rgb]{0.56,0.35,0.01}{\textbf{\textit{#1}}}}
\newcommand{\AttributeTok}[1]{\textcolor[rgb]{0.13,0.29,0.53}{#1}}
\newcommand{\BaseNTok}[1]{\textcolor[rgb]{0.00,0.00,0.81}{#1}}
\newcommand{\BuiltInTok}[1]{#1}
\newcommand{\CharTok}[1]{\textcolor[rgb]{0.31,0.60,0.02}{#1}}
\newcommand{\CommentTok}[1]{\textcolor[rgb]{0.56,0.35,0.01}{\textit{#1}}}
\newcommand{\CommentVarTok}[1]{\textcolor[rgb]{0.56,0.35,0.01}{\textbf{\textit{#1}}}}
\newcommand{\ConstantTok}[1]{\textcolor[rgb]{0.56,0.35,0.01}{#1}}
\newcommand{\ControlFlowTok}[1]{\textcolor[rgb]{0.13,0.29,0.53}{\textbf{#1}}}
\newcommand{\DataTypeTok}[1]{\textcolor[rgb]{0.13,0.29,0.53}{#1}}
\newcommand{\DecValTok}[1]{\textcolor[rgb]{0.00,0.00,0.81}{#1}}
\newcommand{\DocumentationTok}[1]{\textcolor[rgb]{0.56,0.35,0.01}{\textbf{\textit{#1}}}}
\newcommand{\ErrorTok}[1]{\textcolor[rgb]{0.64,0.00,0.00}{\textbf{#1}}}
\newcommand{\ExtensionTok}[1]{#1}
\newcommand{\FloatTok}[1]{\textcolor[rgb]{0.00,0.00,0.81}{#1}}
\newcommand{\FunctionTok}[1]{\textcolor[rgb]{0.13,0.29,0.53}{\textbf{#1}}}
\newcommand{\ImportTok}[1]{#1}
\newcommand{\InformationTok}[1]{\textcolor[rgb]{0.56,0.35,0.01}{\textbf{\textit{#1}}}}
\newcommand{\KeywordTok}[1]{\textcolor[rgb]{0.13,0.29,0.53}{\textbf{#1}}}
\newcommand{\NormalTok}[1]{#1}
\newcommand{\OperatorTok}[1]{\textcolor[rgb]{0.81,0.36,0.00}{\textbf{#1}}}
\newcommand{\OtherTok}[1]{\textcolor[rgb]{0.56,0.35,0.01}{#1}}
\newcommand{\PreprocessorTok}[1]{\textcolor[rgb]{0.56,0.35,0.01}{\textit{#1}}}
\newcommand{\RegionMarkerTok}[1]{#1}
\newcommand{\SpecialCharTok}[1]{\textcolor[rgb]{0.81,0.36,0.00}{\textbf{#1}}}
\newcommand{\SpecialStringTok}[1]{\textcolor[rgb]{0.31,0.60,0.02}{#1}}
\newcommand{\StringTok}[1]{\textcolor[rgb]{0.31,0.60,0.02}{#1}}
\newcommand{\VariableTok}[1]{\textcolor[rgb]{0.00,0.00,0.00}{#1}}
\newcommand{\VerbatimStringTok}[1]{\textcolor[rgb]{0.31,0.60,0.02}{#1}}
\newcommand{\WarningTok}[1]{\textcolor[rgb]{0.56,0.35,0.01}{\textbf{\textit{#1}}}}
\usepackage{longtable,booktabs,array}
\usepackage{calc} % for calculating minipage widths
% Correct order of tables after \paragraph or \subparagraph
\usepackage{etoolbox}
\makeatletter
\patchcmd\longtable{\par}{\if@noskipsec\mbox{}\fi\par}{}{}
\makeatother
% Allow footnotes in longtable head/foot
\IfFileExists{footnotehyper.sty}{\usepackage{footnotehyper}}{\usepackage{footnote}}
\makesavenoteenv{longtable}
\usepackage{graphicx}
\makeatletter
\def\maxwidth{\ifdim\Gin@nat@width>\linewidth\linewidth\else\Gin@nat@width\fi}
\def\maxheight{\ifdim\Gin@nat@height>\textheight\textheight\else\Gin@nat@height\fi}
\makeatother
% Scale images if necessary, so that they will not overflow the page
% margins by default, and it is still possible to overwrite the defaults
% using explicit options in \includegraphics[width, height, ...]{}
\setkeys{Gin}{width=\maxwidth,height=\maxheight,keepaspectratio}
% Set default figure placement to htbp
\makeatletter
\def\fps@figure{htbp}
\makeatother
\setlength{\emergencystretch}{3em} % prevent overfull lines
\providecommand{\tightlist}{%
  \setlength{\itemsep}{0pt}\setlength{\parskip}{0pt}}
\setcounter{secnumdepth}{-\maxdimen} % remove section numbering
\ifLuaTeX
  \usepackage{selnolig}  % disable illegal ligatures
\fi
\IfFileExists{bookmark.sty}{\usepackage{bookmark}}{\usepackage{hyperref}}
\IfFileExists{xurl.sty}{\usepackage{xurl}}{} % add URL line breaks if available
\urlstyle{same}
\hypersetup{
  pdftitle={M2 Lab},
  pdfauthor={YOUR NAME HERE},
  hidelinks,
  pdfcreator={LaTeX via pandoc}}

\title{M2 Lab}
\author{YOUR NAME HERE}
\date{Spring 2024}

\begin{document}
\maketitle

\subsection{Libraries and Functions}\label{libraries-and-functions}

\begin{Shaded}
\begin{Highlighting}[]
\FunctionTok{library}\NormalTok{(tidyverse) }\CommentTok{\# for data manipulation}
\FunctionTok{library}\NormalTok{(psych) }\CommentTok{\# for pairiwse scatter plot function (not necessary but in case you want to use that function from the lecture code)}
\FunctionTok{library}\NormalTok{(car) }\CommentTok{\# for extra model diagnostics like plots and vif() function}
\FunctionTok{library}\NormalTok{(MASS) }\CommentTok{\# for rlm() function (robust regression)}
\FunctionTok{library}\NormalTok{(plotly)}
\FunctionTok{library}\NormalTok{(ggthemr)}
\FunctionTok{ggthemr}\NormalTok{(}\StringTok{"flat dark"}\NormalTok{)}
\FunctionTok{library}\NormalTok{(gridExtra)}
\FunctionTok{library}\NormalTok{(ggh4x)}
\FunctionTok{library}\NormalTok{(kableExtra)}
\FunctionTok{library}\NormalTok{(jtools)}
\FunctionTok{library}\NormalTok{(rlang)}
\FunctionTok{library}\NormalTok{(here)}
\FunctionTok{library}\NormalTok{(skimr)}
\FunctionTok{library}\NormalTok{(corrplot)}
\FunctionTok{library}\NormalTok{(broom)}
\end{Highlighting}
\end{Shaded}

\begin{Shaded}
\begin{Highlighting}[]
\CommentTok{\# Function to summarize variables with optional grouping}
\CommentTok{\# Usage: summarize.var(dataset, var, group\_var (optional))}
\NormalTok{summarize.var }\OtherTok{\textless{}{-}} \ControlFlowTok{function}\NormalTok{(dataset, var, group\_var) \{}
  \FunctionTok{require}\NormalTok{(tidyverse)}
  \FunctionTok{require}\NormalTok{(rlang)}

\NormalTok{  result }\OtherTok{\textless{}{-}}\NormalTok{ dataset }\SpecialCharTok{\%\textgreater{}\%}
    \FunctionTok{group\_by}\NormalTok{(\{\{ group\_var \}\}) }\SpecialCharTok{\%\textgreater{}\%}
    \FunctionTok{summarise}\NormalTok{(}
      \AttributeTok{mean =} \FunctionTok{mean}\NormalTok{(\{\{ var \}\}, }\AttributeTok{na.rm =} \ConstantTok{TRUE}\NormalTok{),}
      \AttributeTok{median =} \FunctionTok{median}\NormalTok{(\{\{ var \}\}, }\AttributeTok{na.rm =} \ConstantTok{TRUE}\NormalTok{),}
      \AttributeTok{min =} \FunctionTok{min}\NormalTok{(\{\{ var \}\}, }\AttributeTok{na.rm =} \ConstantTok{TRUE}\NormalTok{),}
      \AttributeTok{max =} \FunctionTok{max}\NormalTok{(\{\{ var \}\}, }\AttributeTok{na.rm =} \ConstantTok{TRUE}\NormalTok{),}
      \AttributeTok{q25 =} \FunctionTok{quantile}\NormalTok{(\{\{ var \}\}, }\FloatTok{0.25}\NormalTok{, }\AttributeTok{na.rm =} \ConstantTok{TRUE}\NormalTok{), }\CommentTok{\# first quartile}
      \AttributeTok{q75 =} \FunctionTok{quantile}\NormalTok{(\{\{ var \}\}, }\FloatTok{0.75}\NormalTok{, }\AttributeTok{na.rm =} \ConstantTok{TRUE}\NormalTok{), }\CommentTok{\# third quartile}
      \AttributeTok{sd =} \FunctionTok{sd}\NormalTok{(\{\{ var \}\}, }\AttributeTok{na.rm =} \ConstantTok{TRUE}\NormalTok{),}
      \AttributeTok{n =} \FunctionTok{length}\NormalTok{(}\FunctionTok{na.omit}\NormalTok{(\{\{ var \}\}))}
\NormalTok{    ) }\SpecialCharTok{\%\textgreater{}\%}
    \FunctionTok{mutate}\NormalTok{(}\AttributeTok{IQR =}\NormalTok{ q75 }\SpecialCharTok{{-}}\NormalTok{ q25)}

  \FunctionTok{return}\NormalTok{(result)}
\NormalTok{\}}
\end{Highlighting}
\end{Shaded}

\begin{Shaded}
\begin{Highlighting}[]
\FunctionTok{library}\NormalTok{(ggplot2)}

\NormalTok{ggResid }\OtherTok{\textless{}{-}} \ControlFlowTok{function}\NormalTok{(fit) \{}
  \CommentTok{\# Extract fitted values and residuals from the linear model}
\NormalTok{  fitted\_values }\OtherTok{\textless{}{-}}\NormalTok{ fit}\SpecialCharTok{$}\NormalTok{fitted.values}
\NormalTok{  residuals }\OtherTok{\textless{}{-}}\NormalTok{ fit}\SpecialCharTok{$}\NormalTok{residuals}

  \CommentTok{\# Create a data frame for the residuals vs. fitted values plot}
\NormalTok{  resid\_data }\OtherTok{\textless{}{-}} \FunctionTok{data.frame}\NormalTok{(}\AttributeTok{Fitted =}\NormalTok{ fitted\_values, }\AttributeTok{Residuals =}\NormalTok{ residuals)}

  \CommentTok{\# Create the plot}
  \FunctionTok{ggplot}\NormalTok{(resid\_data, }\FunctionTok{aes}\NormalTok{(}\AttributeTok{x =}\NormalTok{ Fitted, }\AttributeTok{y =}\NormalTok{ Residuals)) }\SpecialCharTok{+}
    \FunctionTok{geom\_point}\NormalTok{() }\SpecialCharTok{+}
    \FunctionTok{geom\_hline}\NormalTok{(}\AttributeTok{yintercept =} \DecValTok{0}\NormalTok{, }\AttributeTok{color =} \StringTok{"red"}\NormalTok{, }\AttributeTok{linetype =} \StringTok{"dashed"}\NormalTok{) }\SpecialCharTok{+}
    \FunctionTok{labs}\NormalTok{(}
      \AttributeTok{title =} \StringTok{"Residuals vs. Fitted Values"}\NormalTok{,}
      \AttributeTok{x =} \StringTok{"Fitted Values"}\NormalTok{,}
      \AttributeTok{y =} \StringTok{"Residuals"}
\NormalTok{    )}
\NormalTok{\}}

\CommentTok{\# Example usage:}
\CommentTok{\# fit \textless{}{-} lm(mpg \textasciitilde{} disp + hp, data = mtcars)}
\CommentTok{\# ggResid(fit)}
\end{Highlighting}
\end{Shaded}

\begin{Shaded}
\begin{Highlighting}[]
\CommentTok{\# ggplotRegression}
\CommentTok{\# use to plot linear model with summary statistics and error}
\CommentTok{\# Usage: ggplotRegression(data)}
\NormalTok{ggRegression }\OtherTok{\textless{}{-}} \ControlFlowTok{function}\NormalTok{(fit, }\AttributeTok{title =} \StringTok{"title"}\NormalTok{) \{}
  \FunctionTok{require}\NormalTok{(ggplot2)}

  \FunctionTok{ggplot}\NormalTok{(fit}\SpecialCharTok{$}\NormalTok{model, }\FunctionTok{aes\_string}\NormalTok{(}\AttributeTok{x =} \FunctionTok{names}\NormalTok{(fit}\SpecialCharTok{$}\NormalTok{model)[}\DecValTok{2}\NormalTok{], }\AttributeTok{y =} \FunctionTok{names}\NormalTok{(fit}\SpecialCharTok{$}\NormalTok{model)[}\DecValTok{1}\NormalTok{])) }\SpecialCharTok{+}
    \FunctionTok{geom\_point}\NormalTok{() }\SpecialCharTok{+}
    \FunctionTok{stat\_smooth}\NormalTok{(}
      \AttributeTok{method =} \StringTok{"lm"}\NormalTok{,}
      \AttributeTok{geom =} \StringTok{"smooth"}
\NormalTok{    ) }\SpecialCharTok{+}
    \FunctionTok{labs}\NormalTok{(}
      \AttributeTok{title =}\NormalTok{ title,}
      \AttributeTok{subtitle =} \FunctionTok{paste}\NormalTok{(}
        \StringTok{"Adj R2 = "}\NormalTok{, }\FunctionTok{signif}\NormalTok{(}\FunctionTok{summary}\NormalTok{(fit)}\SpecialCharTok{$}\NormalTok{adj.r.squared, }\DecValTok{5}\NormalTok{),}
        \StringTok{"Intercept ="}\NormalTok{, }\FunctionTok{signif}\NormalTok{(fit}\SpecialCharTok{$}\NormalTok{coef[[}\DecValTok{1}\NormalTok{]], }\DecValTok{5}\NormalTok{),}
        \StringTok{" Slope ="}\NormalTok{, }\FunctionTok{signif}\NormalTok{(fit}\SpecialCharTok{$}\NormalTok{coef[[}\DecValTok{2}\NormalTok{]], }\DecValTok{5}\NormalTok{),}
        \StringTok{" P ="}\NormalTok{, }\FunctionTok{signif}\NormalTok{(}\FunctionTok{summary}\NormalTok{(fit)}\SpecialCharTok{$}\NormalTok{coef[}\DecValTok{2}\NormalTok{, }\DecValTok{4}\NormalTok{], }\DecValTok{5}\NormalTok{)}
\NormalTok{      )}
\NormalTok{    )}
\NormalTok{\}}
\end{Highlighting}
\end{Shaded}

\begin{Shaded}
\begin{Highlighting}[]
\NormalTok{qqTest }\OtherTok{\textless{}{-}} \ControlFlowTok{function}\NormalTok{(data, sample, }\AttributeTok{title =} \StringTok{"title"}\NormalTok{, }\AttributeTok{facet\_var =} \ConstantTok{NULL}\NormalTok{) \{}
  \FunctionTok{require}\NormalTok{(tidyverse)}

\NormalTok{  gg }\OtherTok{\textless{}{-}} \FunctionTok{ggplot}\NormalTok{(}
\NormalTok{    data,}
    \FunctionTok{aes}\NormalTok{(}\AttributeTok{sample =}\NormalTok{ \{\{ sample \}\})}
\NormalTok{  ) }\SpecialCharTok{+}
    \FunctionTok{stat\_qq}\NormalTok{() }\SpecialCharTok{+}
    \FunctionTok{stat\_qq\_line}\NormalTok{(}
      \AttributeTok{size =} \FloatTok{0.8}\NormalTok{,}
\NormalTok{    ) }\SpecialCharTok{+}
    \FunctionTok{labs}\NormalTok{(}
      \AttributeTok{title =}\NormalTok{ title,}
      \AttributeTok{x =} \StringTok{"Theoretical Quantiles"}\NormalTok{,}
      \AttributeTok{y =} \StringTok{"Sample Quantiles"}
\NormalTok{    ) }\SpecialCharTok{+}
    \FunctionTok{guides}\NormalTok{(}\AttributeTok{color =} \StringTok{"none"}\NormalTok{)}

  \ControlFlowTok{if}\NormalTok{ (}\SpecialCharTok{!}\FunctionTok{is.null}\NormalTok{(facet\_var)) \{}
    \CommentTok{\# Extract colors from facet\_var variable}
\NormalTok{    colors }\OtherTok{\textless{}{-}} \FunctionTok{pull}\NormalTok{(data, \{\{ facet\_var \}\})}

\NormalTok{    gg }\OtherTok{\textless{}{-}}\NormalTok{ gg }\SpecialCharTok{+} \FunctionTok{aes}\NormalTok{(}\AttributeTok{color =} \FunctionTok{factor}\NormalTok{(colors)) }\SpecialCharTok{+} \FunctionTok{facet\_wrap}\NormalTok{(}\FunctionTok{as.formula}\NormalTok{(}\FunctionTok{paste}\NormalTok{(}\StringTok{"\textasciitilde{}"}\NormalTok{, facet\_var)))}
\NormalTok{  \}}

  \FunctionTok{return}\NormalTok{(gg)}
\NormalTok{\}}
\end{Highlighting}
\end{Shaded}

\begin{Shaded}
\begin{Highlighting}[]
\CommentTok{\# density plots with optional grouping variable}
\CommentTok{\# can be faceted or together by specifying either group or color}
\CommentTok{\# usage: density.dodge(data, sample, group = "group" (optional), color = "NULL" (optional), title = "title" (optional) )}
\NormalTok{ggDensity }\OtherTok{\textless{}{-}} \ControlFlowTok{function}\NormalTok{(data, sample, }\AttributeTok{group =} \ConstantTok{NULL}\NormalTok{, }\AttributeTok{color =} \ConstantTok{NULL}\NormalTok{, }\AttributeTok{title =} \StringTok{"Insert Title"}\NormalTok{) \{}
  \FunctionTok{require}\NormalTok{(ggplot2)}

\NormalTok{  plot }\OtherTok{\textless{}{-}} \FunctionTok{ggplot}\NormalTok{(data, }\FunctionTok{aes}\NormalTok{(}\AttributeTok{x =}\NormalTok{ \{\{ sample \}\})) }\SpecialCharTok{+}
    \FunctionTok{geom\_density}\NormalTok{() }\SpecialCharTok{+}
    \FunctionTok{labs}\NormalTok{(}
      \AttributeTok{title =}\NormalTok{ title,}
      \AttributeTok{y =} \StringTok{"Frequency"}\NormalTok{,}
      \AttributeTok{color =} \ConstantTok{NULL}
\NormalTok{    )}

  \ControlFlowTok{if}\NormalTok{ (}\SpecialCharTok{!}\FunctionTok{is.null}\NormalTok{(group)) \{}
\NormalTok{    colors }\OtherTok{\textless{}{-}} \FunctionTok{pull}\NormalTok{(data, \{\{ group \}\})}

\NormalTok{    plot }\OtherTok{\textless{}{-}}\NormalTok{ plot }\SpecialCharTok{+} \FunctionTok{aes}\NormalTok{(}\AttributeTok{color =} \FunctionTok{factor}\NormalTok{(colors)) }\SpecialCharTok{+}
      \FunctionTok{facet\_wrap}\NormalTok{(}\FunctionTok{as.formula}\NormalTok{(}\FunctionTok{paste}\NormalTok{(}\StringTok{"\textasciitilde{}"}\NormalTok{, group))) }\SpecialCharTok{+}
      \FunctionTok{labs}\NormalTok{(}\AttributeTok{color =} \StringTok{"Group"}\NormalTok{)}
\NormalTok{  \}}

  \ControlFlowTok{if}\NormalTok{ (}\SpecialCharTok{!}\FunctionTok{is.null}\NormalTok{(color)) \{}
\NormalTok{    colors }\OtherTok{\textless{}{-}} \FunctionTok{pull}\NormalTok{(data, \{\{ color \}\})}

\NormalTok{    plot }\OtherTok{\textless{}{-}}\NormalTok{ plot }\SpecialCharTok{+} \FunctionTok{aes}\NormalTok{(}\AttributeTok{fill =} \FunctionTok{factor}\NormalTok{(colors), }\AttributeTok{color =} \FunctionTok{factor}\NormalTok{(colors), }\AttributeTok{alpha =} \FloatTok{0.02}\NormalTok{) }\SpecialCharTok{+}
      \FunctionTok{labs}\NormalTok{(}\AttributeTok{fill =} \StringTok{"Group"}\NormalTok{) }\SpecialCharTok{+}
      \FunctionTok{geom\_density}\NormalTok{(}\AttributeTok{size =} \DecValTok{1}\NormalTok{) }\SpecialCharTok{+}
      \FunctionTok{guides}\NormalTok{(}\AttributeTok{color =} \StringTok{"none"}\NormalTok{, }\AttributeTok{alpha =} \StringTok{"none"}\NormalTok{)}
\NormalTok{  \}}

  \FunctionTok{return}\NormalTok{(plot)}
\NormalTok{\}}
\end{Highlighting}
\end{Shaded}

\section{Data Management}\label{data-management}

The following data come from a study of the impacts of toluene exposure
through inhalation on toluene concentration in the blood. In the study,
60 rats were exposed to differing levels of toluene inhalation, from
11.34 to 1744.7 parts per million (ppm) (newppm) for a duration of 3
hours. Blood levels were measured (bloodtol) following exposure measured
in mg/L. Other variables measured were weight in grams (weight), age in
days (age), and snout size as either short (snoutsize=1) or long
(snoutsize=2).

We will use these data to model the relationship between toluene
exposure and blood concentration, possibly adjusting for other
variables.

\begin{Shaded}
\begin{Highlighting}[]
\FunctionTok{setwd}\NormalTok{(}\StringTok{"Lab2"}\NormalTok{)}
\end{Highlighting}
\end{Shaded}

\begin{verbatim}
## Error in setwd("Lab2"): cannot change working directory
\end{verbatim}

\begin{Shaded}
\begin{Highlighting}[]
\CommentTok{\# reading data and making factor variable for snout size}
\NormalTok{data }\OtherTok{\textless{}{-}} \FunctionTok{read\_csv}\NormalTok{(}\StringTok{"hwdata2.csv"}\NormalTok{) }\SpecialCharTok{\%\textgreater{}\%}
  \FunctionTok{mutate}\NormalTok{(}
    \AttributeTok{snout\_f =} \FunctionTok{factor}\NormalTok{(snoutsize }\SpecialCharTok{{-}} \DecValTok{1}\NormalTok{,}
      \AttributeTok{levels =} \FunctionTok{c}\NormalTok{(}\DecValTok{0}\NormalTok{, }\DecValTok{1}\NormalTok{),}
      \AttributeTok{labels =} \FunctionTok{c}\NormalTok{(}\StringTok{"short"}\NormalTok{, }\StringTok{"long"}\NormalTok{)}
\NormalTok{    ),}
    \AttributeTok{snoutsize =} \FunctionTok{as.factor}\NormalTok{(snoutsize)}
\NormalTok{  )}
\end{Highlighting}
\end{Shaded}

\section{Part 1) Visualize data}\label{part-1-visualize-data}

\textbf{Calculate some descriptive statistics and display some plots to
learn about the data.} (You don't have to do extensive summaries/plots,
but get a sense of the types of variables, any missingness, distribution
of certain variables that could be important in the regression models,
and relationships between variables). You should create a factor
variable for snout size.

\begin{Shaded}
\begin{Highlighting}[]
\NormalTok{weightStats }\OtherTok{\textless{}{-}} \FunctionTok{summarize.var}\NormalTok{(data, weight, snout\_f) }\SpecialCharTok{\%\textgreater{}\%}
  \FunctionTok{mutate\_if}\NormalTok{(is.numeric, round, }\AttributeTok{digits =} \DecValTok{2}\NormalTok{)}
\NormalTok{ageStats }\OtherTok{\textless{}{-}} \FunctionTok{summarize.var}\NormalTok{(data, age, snout\_f) }\SpecialCharTok{\%\textgreater{}\%}
  \FunctionTok{mutate\_if}\NormalTok{(is.numeric, round, }\AttributeTok{digits =} \DecValTok{2}\NormalTok{)}
\NormalTok{bloodStats }\OtherTok{\textless{}{-}} \FunctionTok{summarize.var}\NormalTok{(data, bloodtol, snout\_f) }\SpecialCharTok{\%\textgreater{}\%}
  \FunctionTok{mutate\_if}\NormalTok{(is.numeric, round, }\AttributeTok{digits =} \DecValTok{2}\NormalTok{)}
\NormalTok{ppmStats }\OtherTok{\textless{}{-}} \FunctionTok{summarize.var}\NormalTok{(data, newppm, snout\_f) }\SpecialCharTok{\%\textgreater{}\%}
  \FunctionTok{mutate\_if}\NormalTok{(is.numeric, round, }\AttributeTok{digits =} \DecValTok{2}\NormalTok{)}

\NormalTok{weightplot }\OtherTok{\textless{}{-}} \FunctionTok{ggDensity}\NormalTok{(data, weight, }\AttributeTok{color =} \StringTok{"snout\_f"}\NormalTok{, }\AttributeTok{title =} \StringTok{"1a. Weight"}\NormalTok{) }\SpecialCharTok{+} \FunctionTok{geom\_vline}\NormalTok{(}\FunctionTok{aes}\NormalTok{(}\AttributeTok{xintercept =}\NormalTok{ mean, }\AttributeTok{color =}\NormalTok{ snout\_f), }\AttributeTok{data =}\NormalTok{ weightStats, }\AttributeTok{linetype =} \StringTok{"dashed"}\NormalTok{)}
\NormalTok{ageplot }\OtherTok{\textless{}{-}} \FunctionTok{ggDensity}\NormalTok{(data, age, }\AttributeTok{color =} \StringTok{"snout\_f"}\NormalTok{, }\AttributeTok{title =} \StringTok{"1b. Age"}\NormalTok{) }\SpecialCharTok{+} \FunctionTok{geom\_vline}\NormalTok{(}\FunctionTok{aes}\NormalTok{(}\AttributeTok{xintercept =}\NormalTok{ mean, }\AttributeTok{color =}\NormalTok{ snout\_f), }\AttributeTok{data =}\NormalTok{ ageStats, }\AttributeTok{linetype =} \StringTok{"dashed"}\NormalTok{)}
\NormalTok{bloodplot }\OtherTok{\textless{}{-}} \FunctionTok{ggDensity}\NormalTok{(data, bloodtol, }\AttributeTok{color =} \StringTok{"snout\_f"}\NormalTok{, }\AttributeTok{title =} \StringTok{"1c. Blood Toluene (mg/L)"}\NormalTok{) }\SpecialCharTok{+} \FunctionTok{geom\_vline}\NormalTok{(}\FunctionTok{aes}\NormalTok{(}\AttributeTok{xintercept =}\NormalTok{ mean, }\AttributeTok{color =}\NormalTok{ snout\_f), }\AttributeTok{data =}\NormalTok{ bloodStats, }\AttributeTok{linetype =} \StringTok{"dashed"}\NormalTok{)}
\NormalTok{newppmplot }\OtherTok{\textless{}{-}} \FunctionTok{ggDensity}\NormalTok{(data, newppm, }\AttributeTok{color =} \StringTok{"snout\_f"}\NormalTok{, }\AttributeTok{title =} \StringTok{"1d. Toluene Exposure (ppm)"}\NormalTok{) }\SpecialCharTok{+} \FunctionTok{geom\_vline}\NormalTok{(}\FunctionTok{aes}\NormalTok{(}\AttributeTok{xintercept =}\NormalTok{ mean, }\AttributeTok{color =}\NormalTok{ snout\_f), }\AttributeTok{data =}\NormalTok{ ppmStats, }\AttributeTok{linetype =} \StringTok{"dashed"}\NormalTok{)}

\FunctionTok{grid.arrange}\NormalTok{(}
\NormalTok{  weightplot, ageplot, bloodplot, newppmplot,}
  \AttributeTok{nrow =} \DecValTok{2}\NormalTok{, }\AttributeTok{ncol =} \DecValTok{2}\NormalTok{,}
  \AttributeTok{top =} \StringTok{"Figure 1. Density Plots of Continuous Variables by Snout Size"}
\NormalTok{)}
\end{Highlighting}
\end{Shaded}

\includegraphics{Lab2_files/figure-latex/unnamed-chunk-9-1.pdf}

\begin{Shaded}
\begin{Highlighting}[]
\CommentTok{\# ggpubr::ggdensity(data, x = "weight", add = "mean", rug = TRUE, color = "snout\_f", fill = "snout\_f") \%\textgreater{}\% ggplotly()}

\CommentTok{\# pWeight \textless{}{-} ggplotly(weightplot)}
\CommentTok{\# pAge \textless{}{-} ggplotly(ageplot)}
\CommentTok{\# pBlood \textless{}{-} ggplotly(bloodplot)}
\CommentTok{\# pPPM \textless{}{-} ggplotly(newppmplot)}

\CommentTok{\# subplot(pWeight, pAge, pBlood, pPPM, nrows = 2)}

\NormalTok{skimr}\SpecialCharTok{::}\FunctionTok{skim}\NormalTok{(data)}
\end{Highlighting}
\end{Shaded}

\begin{longtable}[]{@{}ll@{}}
\caption{Data summary}\tabularnewline
\toprule\noalign{}
\endfirsthead
\endhead
\bottomrule\noalign{}
\endlastfoot
Name & data \\
Number of rows & 60 \\
Number of columns & 7 \\
\_\_\_\_\_\_\_\_\_\_\_\_\_\_\_\_\_\_\_\_\_\_\_ & \\
Column type frequency: & \\
factor & 2 \\
numeric & 5 \\
\_\_\_\_\_\_\_\_\_\_\_\_\_\_\_\_\_\_\_\_\_\_\_\_ & \\
Group variables & None \\
\end{longtable}

\textbf{Variable type: factor}

\begin{longtable}[]{@{}
  >{\raggedright\arraybackslash}p{(\columnwidth - 10\tabcolsep) * \real{0.1944}}
  >{\raggedleft\arraybackslash}p{(\columnwidth - 10\tabcolsep) * \real{0.1389}}
  >{\raggedleft\arraybackslash}p{(\columnwidth - 10\tabcolsep) * \real{0.1944}}
  >{\raggedright\arraybackslash}p{(\columnwidth - 10\tabcolsep) * \real{0.1111}}
  >{\raggedleft\arraybackslash}p{(\columnwidth - 10\tabcolsep) * \real{0.1250}}
  >{\raggedright\arraybackslash}p{(\columnwidth - 10\tabcolsep) * \real{0.2361}}@{}}
\toprule\noalign{}
\begin{minipage}[b]{\linewidth}\raggedright
skim\_variable
\end{minipage} & \begin{minipage}[b]{\linewidth}\raggedleft
n\_missing
\end{minipage} & \begin{minipage}[b]{\linewidth}\raggedleft
complete\_rate
\end{minipage} & \begin{minipage}[b]{\linewidth}\raggedright
ordered
\end{minipage} & \begin{minipage}[b]{\linewidth}\raggedleft
n\_unique
\end{minipage} & \begin{minipage}[b]{\linewidth}\raggedright
top\_counts
\end{minipage} \\
\midrule\noalign{}
\endhead
\bottomrule\noalign{}
\endlastfoot
snoutsize & 0 & 1 & FALSE & 2 & 2: 45, 1: 15 \\
snout\_f & 0 & 1 & FALSE & 2 & lon: 45, sho: 15 \\
\end{longtable}

\textbf{Variable type: numeric}

\begin{longtable}[]{@{}
  >{\raggedright\arraybackslash}p{(\columnwidth - 20\tabcolsep) * \real{0.1085}}
  >{\raggedleft\arraybackslash}p{(\columnwidth - 20\tabcolsep) * \real{0.0775}}
  >{\raggedleft\arraybackslash}p{(\columnwidth - 20\tabcolsep) * \real{0.1085}}
  >{\raggedleft\arraybackslash}p{(\columnwidth - 20\tabcolsep) * \real{0.0543}}
  >{\raggedleft\arraybackslash}p{(\columnwidth - 20\tabcolsep) * \real{0.0543}}
  >{\raggedleft\arraybackslash}p{(\columnwidth - 20\tabcolsep) * \real{0.0543}}
  >{\raggedleft\arraybackslash}p{(\columnwidth - 20\tabcolsep) * \real{0.0543}}
  >{\raggedleft\arraybackslash}p{(\columnwidth - 20\tabcolsep) * \real{0.0543}}
  >{\raggedleft\arraybackslash}p{(\columnwidth - 20\tabcolsep) * \real{0.0543}}
  >{\raggedleft\arraybackslash}p{(\columnwidth - 20\tabcolsep) * \real{0.0620}}
  >{\raggedright\arraybackslash}p{(\columnwidth - 20\tabcolsep) * \real{0.3178}}@{}}
\toprule\noalign{}
\begin{minipage}[b]{\linewidth}\raggedright
skim\_variable
\end{minipage} & \begin{minipage}[b]{\linewidth}\raggedleft
n\_missing
\end{minipage} & \begin{minipage}[b]{\linewidth}\raggedleft
complete\_rate
\end{minipage} & \begin{minipage}[b]{\linewidth}\raggedleft
mean
\end{minipage} & \begin{minipage}[b]{\linewidth}\raggedleft
sd
\end{minipage} & \begin{minipage}[b]{\linewidth}\raggedleft
p0
\end{minipage} & \begin{minipage}[b]{\linewidth}\raggedleft
p25
\end{minipage} & \begin{minipage}[b]{\linewidth}\raggedleft
p50
\end{minipage} & \begin{minipage}[b]{\linewidth}\raggedleft
p75
\end{minipage} & \begin{minipage}[b]{\linewidth}\raggedleft
p100
\end{minipage} & \begin{minipage}[b]{\linewidth}\raggedright
hist
\end{minipage} \\
\midrule\noalign{}
\endhead
\bottomrule\noalign{}
\endlastfoot
rat & 0 & 1 & 30.50 & 17.46 & 1.00 & 15.75 & 30.50 & 45.25 & 60.00 &
▇▇▇▇▇ \\
bloodtol & 0 & 1 & 11.00 & 12.62 & 0.22 & 0.70 & 6.54 & 22.07 & 39.54 &
▇▃▁▂▂ \\
weight & 0 & 1 & 380.55 & 72.71 & 100.00 & 350.25 & 373.50 & 406.00 &
700.00 & ▁▂▇▁▁ \\
age & 0 & 1 & 83.40 & 5.70 & 70.00 & 82.25 & 85.00 & 86.00 & 95.00 &
▃▂▇▅▂ \\
newppm & 0 & 1 & 292.48 & 311.99 & 3.57 & 50.70 & 104.50 & 502.50 &
1000.00 & ▇▂▂▂▂ \\
\end{longtable}

\begin{Shaded}
\begin{Highlighting}[]
\FunctionTok{pairs.panels}\NormalTok{(data[, }\FunctionTok{c}\NormalTok{(}\StringTok{"bloodtol"}\NormalTok{, }\StringTok{"weight"}\NormalTok{, }\StringTok{"age"}\NormalTok{, }\StringTok{"newppm"}\NormalTok{)],}
  \AttributeTok{ellipses =} \ConstantTok{FALSE}\NormalTok{, }\AttributeTok{density =} \ConstantTok{FALSE}\NormalTok{, }\AttributeTok{hist.col =} \StringTok{"deepskyblue"}
\NormalTok{)}
\end{Highlighting}
\end{Shaded}

\includegraphics{Lab2_files/figure-latex/unnamed-chunk-9-2.pdf}

Much of the data appear to be skewed and multimodal, although weight is
much less so than the other continuous variables. There were no missing
values.

\section{Part 2) SLR}\label{part-2-slr}

\textbf{Fit a regression model for blood toluene (bloodtol) that
contains only toluene exposure (newppm) as a predictor and perform a
graphical analysis to assess the linearity, equal variance, and
normality assumptions of linear regression.} \textbf{Interpret the plots
and evaluate if you believe the assumptions are met.}

\begin{Shaded}
\begin{Highlighting}[]
\CommentTok{\# fit slr model}
\NormalTok{bloodPPM }\OtherTok{\textless{}{-}} \FunctionTok{lm}\NormalTok{(bloodtol }\SpecialCharTok{\textasciitilde{}}\NormalTok{ newppm, }\AttributeTok{data =}\NormalTok{ data)}

\FunctionTok{ggRegression}\NormalTok{(bloodPPM, }\AttributeTok{title =} \StringTok{"Plot of Blood Toluene by Toluene Exposure (PPM)"}\NormalTok{)}
\end{Highlighting}
\end{Shaded}

\begin{verbatim}
## `geom_smooth()` using formula = 'y ~ x'
\end{verbatim}

\includegraphics{Lab2_files/figure-latex/unnamed-chunk-10-1.pdf}

\begin{Shaded}
\begin{Highlighting}[]
\FunctionTok{ggResid}\NormalTok{(bloodPPM)}
\end{Highlighting}
\end{Shaded}

\includegraphics{Lab2_files/figure-latex/unnamed-chunk-10-2.pdf}

\begin{Shaded}
\begin{Highlighting}[]
\FunctionTok{qqTest}\NormalTok{(bloodPPM, .stdresid, }\AttributeTok{title =} \StringTok{"QQ Test of Residuals"}\NormalTok{)}
\end{Highlighting}
\end{Shaded}

\includegraphics{Lab2_files/figure-latex/unnamed-chunk-10-3.pdf}

\begin{Shaded}
\begin{Highlighting}[]
\CommentTok{\# plot residual vs fitted and residual qq{-}plot}
\CommentTok{\# hint: if you\textquotesingle{}ve saved your model above you can use the plot() function and specify which = 1:2 to tell R to only print the first 2 plots}
\end{Highlighting}
\end{Shaded}

Interpretation of plots/evaluation of assumptions:\\
The plot of the residuals shows a pattern, especially at the beginning,
rather than being random-appearing, suggesting potential violation of
linearity. The spread of residuals also appears to further deviate from
the line as the fitted values go up, suggesting possible unequal
variance. The QQ-plot of the residuals suggests that they may not be
normally distributed and have heavy tails.

\section{Part 3) MLR}\label{part-3-mlr}

\textbf{We will now add the other covariates to the model. Assess which
variables appear to be important and explicitly} \textbf{test if there
is effect modification (interaction) of toluene exposure by weight, age,
or snout size.}

\begin{Shaded}
\begin{Highlighting}[]
\CommentTok{\# model with interactions included}
\NormalTok{full\_model }\OtherTok{\textless{}{-}} \FunctionTok{lm}\NormalTok{(bloodtol }\SpecialCharTok{\textasciitilde{}}\NormalTok{ newppm }\SpecialCharTok{+}\NormalTok{ weight }\SpecialCharTok{+}\NormalTok{ age }\SpecialCharTok{+}\NormalTok{ snout\_f, }\AttributeTok{data =}\NormalTok{ data)}
\CommentTok{\# test if the interactions are useful (you can just tun one test to test them all at once)}
\FunctionTok{summary}\NormalTok{(full\_model)}
\end{Highlighting}
\end{Shaded}

\begin{verbatim}
## 
## Call:
## lm(formula = bloodtol ~ newppm + weight + age + snout_f, data = data)
## 
## Residuals:
##      Min       1Q   Median       3Q      Max 
## -21.2982  -2.5936  -0.6433   2.0629  22.7613 
## 
## Coefficients:
##               Estimate Std. Error t value Pr(>|t|)    
## (Intercept) -25.190637  14.283095  -1.764   0.0833 .  
## newppm        0.033114   0.002732  12.122   <2e-16 ***
## weight        0.007849   0.013948   0.563   0.5759    
## age           0.312591   0.150875   2.072   0.0430 *  
## snout_flong  -3.397497   2.274375  -1.494   0.1409    
## ---
## Signif. codes:  0 '***' 0.001 '**' 0.01 '*' 0.05 '.' 0.1 ' ' 1
## 
## Residual standard error: 6.201 on 55 degrees of freedom
## Multiple R-squared:  0.7749, Adjusted R-squared:  0.7585 
## F-statistic: 47.32 on 4 and 55 DF,  p-value: < 2.2e-16
\end{verbatim}

\begin{Shaded}
\begin{Highlighting}[]
\NormalTok{interaction\_model }\OtherTok{\textless{}{-}} \FunctionTok{lm}\NormalTok{(bloodtol }\SpecialCharTok{\textasciitilde{}}\NormalTok{ newppm }\SpecialCharTok{*}\NormalTok{ weight }\SpecialCharTok{+}\NormalTok{ newppm }\SpecialCharTok{*}\NormalTok{ age }\SpecialCharTok{+}\NormalTok{ newppm }\SpecialCharTok{*}\NormalTok{ snout\_f, }\AttributeTok{data =}\NormalTok{ data)}
\FunctionTok{summary}\NormalTok{(interaction\_model)}
\end{Highlighting}
\end{Shaded}

\begin{verbatim}
## 
## Call:
## lm(formula = bloodtol ~ newppm * weight + newppm * age + newppm * 
##     snout_f, data = data)
## 
## Residuals:
##      Min       1Q   Median       3Q      Max 
## -18.4224  -3.2132  -0.5186   1.6325  23.0298 
## 
## Coefficients:
##                      Estimate Std. Error t value Pr(>|t|)
## (Intercept)        -8.030e+00  1.924e+01  -0.417    0.678
## newppm             -4.737e-02  5.771e-02  -0.821    0.415
## weight             -6.331e-04  1.822e-02  -0.035    0.972
## age                 1.354e-01  2.013e-01   0.673    0.504
## snout_flong        -2.440e+00  3.423e+00  -0.713    0.479
## newppm:weight       4.064e-05  4.522e-05   0.899    0.373
## newppm:age          8.146e-04  5.727e-04   1.422    0.161
## newppm:snout_flong -5.339e-03  7.880e-03  -0.678    0.501
## 
## Residual standard error: 6.251 on 52 degrees of freedom
## Multiple R-squared:  0.7837, Adjusted R-squared:  0.7546 
## F-statistic: 26.92 on 7 and 52 DF,  p-value: 3.498e-15
\end{verbatim}

\begin{Shaded}
\begin{Highlighting}[]
\CommentTok{\# anova\_comparison \textless{}{-} anova(full\_model, interaction\_model)}
\CommentTok{\# summary(anova\_comparison)}
\end{Highlighting}
\end{Shaded}

Comment on important variables and if interactions seem necessary:\\
In the full model without interaction terms, newppm and age were
identified as statistically significant predictors of blood tol levels,
with p-values significantly below the 0.05 threshold, indicating their
importance as covariates. However, in the interaction model, none of the
interaction terms between newppm and weight, newppm and age, or newppm
and snout\_flong were statistically significant, as their p-values did
not fall below the significance threshold. This suggests that there is
insufficient evidence to support the presence of effect modification by
these variables on the relationship between toluene exposure (newppm)
and blood tol levels.

\section{Part 4) Collinearity}\label{part-4-collinearity}

\textbf{Check for collinearity between the 4 covariates (exposure, age,
weight, snout size).} \textbf{Are you concerned that collinearity could
be affecting our statistical inference in these data?}

\begin{Shaded}
\begin{Highlighting}[]
\CommentTok{\# measure of collinearity in the model involving exposure, weight, age, and snout size}
\FunctionTok{vif}\NormalTok{(full\_model)}
\end{Highlighting}
\end{Shaded}

\begin{verbatim}
##   newppm   weight      age  snout_f 
## 1.114512 1.578022 1.135562 1.513262
\end{verbatim}

\begin{Shaded}
\begin{Highlighting}[]
\CommentTok{\# also testing a model without weight and snout length (just the significant variables from the partial F{-}test)}
\NormalTok{newppmAge }\OtherTok{\textless{}{-}} \FunctionTok{lm}\NormalTok{(bloodtol }\SpecialCharTok{\textasciitilde{}}\NormalTok{ newppm }\SpecialCharTok{+}\NormalTok{ age, }\AttributeTok{data =}\NormalTok{ data)}
\FunctionTok{summary}\NormalTok{(newppmAge)}
\end{Highlighting}
\end{Shaded}

\begin{verbatim}
## 
## Call:
## lm(formula = bloodtol ~ newppm + age, data = data)
## 
## Residuals:
##      Min       1Q   Median       3Q      Max 
## -22.7955  -2.7982  -0.9977   1.8156  25.0754 
## 
## Coefficients:
##               Estimate Std. Error t value Pr(>|t|)    
## (Intercept) -22.789938  12.003603  -1.899   0.0627 .  
## newppm        0.033909   0.002655  12.771   <2e-16 ***
## age           0.286279   0.145273   1.971   0.0536 .  
## ---
## Signif. codes:  0 '***' 0.001 '**' 0.01 '*' 0.05 '.' 0.1 ' ' 1
## 
## Residual standard error: 6.221 on 57 degrees of freedom
## Multiple R-squared:  0.7652, Adjusted R-squared:  0.757 
## F-statistic: 92.88 on 2 and 57 DF,  p-value: < 2.2e-16
\end{verbatim}

\begin{Shaded}
\begin{Highlighting}[]
\FunctionTok{vif}\NormalTok{(newppmAge)}
\end{Highlighting}
\end{Shaded}

\begin{verbatim}
##   newppm      age 
## 1.046181 1.046181
\end{verbatim}

\begin{Shaded}
\begin{Highlighting}[]
\CommentTok{\# their coefficients did not change much}

\CommentTok{\# making a correlation matrix}
\NormalTok{cor\_matrix }\OtherTok{\textless{}{-}} \FunctionTok{cor}\NormalTok{(data[, }\FunctionTok{c}\NormalTok{(}\StringTok{"newppm"}\NormalTok{, }\StringTok{"weight"}\NormalTok{, }\StringTok{"age"}\NormalTok{)], }\AttributeTok{use =} \StringTok{"complete.obs"}\NormalTok{)}

\FunctionTok{corrplot}\NormalTok{(cor\_matrix,}
  \AttributeTok{method =} \StringTok{"circle"}\NormalTok{, }\AttributeTok{type =} \StringTok{"upper"}\NormalTok{, }\AttributeTok{order =} \StringTok{"hclust"}\NormalTok{,}
  \AttributeTok{tl.col =} \StringTok{"black"}\NormalTok{, }\AttributeTok{tl.srt =} \DecValTok{45}\NormalTok{, }\AttributeTok{addCoef.col =} \StringTok{"black"}
\NormalTok{)}
\end{Highlighting}
\end{Shaded}

\includegraphics{Lab2_files/figure-latex/unnamed-chunk-12-1.pdf}

\textbf{Interpretation:}\\
Despite the results from the full and interaction models (insignificant
partial F-tests) from the previous part, the VIF test of the full model
returns small values. Based on this, there is not enough evidence to
conclude that collinearity is significantly affecting our inference.\\
As found earlier in the lab,

\section{Part 5) Choose a Model}\label{part-5-choose-a-model}

\textbf{Based on your analysis so far, choose the model that you believe
is the best. Use this as your final model. Justify why you chose this
model and comment on any potential assumption violations in this model
(a few residual plots may be helpful for this).}

\begin{Shaded}
\begin{Highlighting}[]
\CommentTok{\# Residual vs. Fitted Plot}
\FunctionTok{ggplot}\NormalTok{(full\_model, }\FunctionTok{aes}\NormalTok{(.fitted, .resid)) }\SpecialCharTok{+}
  \FunctionTok{geom\_point}\NormalTok{() }\SpecialCharTok{+}
  \FunctionTok{geom\_hline}\NormalTok{(}\AttributeTok{yintercept =} \DecValTok{0}\NormalTok{, }\AttributeTok{linetype =} \StringTok{"dashed"}\NormalTok{) }\SpecialCharTok{+}
  \FunctionTok{labs}\NormalTok{(}\AttributeTok{x =} \StringTok{"Fitted Values"}\NormalTok{, }\AttributeTok{y =} \StringTok{"Residuals"}\NormalTok{, }\AttributeTok{title =} \StringTok{"Residual vs. Fitted Plot"}\NormalTok{)}
\end{Highlighting}
\end{Shaded}

\includegraphics{Lab2_files/figure-latex/unnamed-chunk-13-1.pdf}

\begin{Shaded}
\begin{Highlighting}[]
\CommentTok{\# QQ{-}plot of residuals}
\FunctionTok{ggplot}\NormalTok{(full\_model, }\FunctionTok{aes}\NormalTok{(}\AttributeTok{sample =}\NormalTok{ .stdresid)) }\SpecialCharTok{+}
  \FunctionTok{stat\_qq}\NormalTok{() }\SpecialCharTok{+}
  \FunctionTok{stat\_qq\_line}\NormalTok{() }\SpecialCharTok{+}
  \FunctionTok{labs}\NormalTok{(}\AttributeTok{title =} \StringTok{"Normal Q{-}Q Plot"}\NormalTok{)}
\end{Highlighting}
\end{Shaded}

\includegraphics{Lab2_files/figure-latex/unnamed-chunk-13-2.pdf}

\begin{Shaded}
\begin{Highlighting}[]
\CommentTok{\# scale location plot}
\FunctionTok{ggplot}\NormalTok{(full\_model, }\FunctionTok{aes}\NormalTok{(.fitted, }\FunctionTok{sqrt}\NormalTok{(}\FunctionTok{abs}\NormalTok{(.stdresid)))) }\SpecialCharTok{+}
  \FunctionTok{geom\_point}\NormalTok{() }\SpecialCharTok{+}
  \FunctionTok{geom\_smooth}\NormalTok{(}\AttributeTok{se =} \ConstantTok{FALSE}\NormalTok{) }\SpecialCharTok{+}
  \FunctionTok{labs}\NormalTok{(}\AttributeTok{x =} \StringTok{"Fitted Values"}\NormalTok{, }\AttributeTok{y =} \StringTok{"Sqrt(|Standardized Residuals|)"}\NormalTok{, }\AttributeTok{title =} \StringTok{"Scale{-}Location Plot"}\NormalTok{)}
\end{Highlighting}
\end{Shaded}

\begin{verbatim}
## `geom_smooth()` using method = 'loess' and formula = 'y ~ x'
\end{verbatim}

\includegraphics{Lab2_files/figure-latex/unnamed-chunk-13-3.pdf}

\begin{Shaded}
\begin{Highlighting}[]
\CommentTok{\# doing a log transform on the full model}
\NormalTok{logFull }\OtherTok{\textless{}{-}} \FunctionTok{lm}\NormalTok{(}\FunctionTok{log}\NormalTok{(bloodtol) }\SpecialCharTok{\textasciitilde{}}\NormalTok{ newppm }\SpecialCharTok{+}\NormalTok{ weight }\SpecialCharTok{+}\NormalTok{ age }\SpecialCharTok{+}\NormalTok{ snout\_f, }\AttributeTok{data =}\NormalTok{ data)}

\CommentTok{\# qq{-}plot}
\FunctionTok{ggplot}\NormalTok{(logFull, }\FunctionTok{aes}\NormalTok{(}\AttributeTok{sample =}\NormalTok{ .stdresid)) }\SpecialCharTok{+}
  \FunctionTok{stat\_qq}\NormalTok{() }\SpecialCharTok{+}
  \FunctionTok{stat\_qq\_line}\NormalTok{() }\SpecialCharTok{+}
  \FunctionTok{labs}\NormalTok{(}\AttributeTok{title =} \StringTok{"Normal Q{-}Q Plot (log transform)"}\NormalTok{)}
\end{Highlighting}
\end{Shaded}

\includegraphics{Lab2_files/figure-latex/unnamed-chunk-13-4.pdf}

\begin{Shaded}
\begin{Highlighting}[]
\CommentTok{\# residuals plot}
\FunctionTok{ggplot}\NormalTok{(logFull, }\FunctionTok{aes}\NormalTok{(.fitted, .resid)) }\SpecialCharTok{+}
  \FunctionTok{geom\_point}\NormalTok{() }\SpecialCharTok{+}
  \FunctionTok{geom\_hline}\NormalTok{(}\AttributeTok{yintercept =} \DecValTok{0}\NormalTok{, }\AttributeTok{linetype =} \StringTok{"dashed"}\NormalTok{) }\SpecialCharTok{+}
  \FunctionTok{labs}\NormalTok{(}\AttributeTok{x =} \StringTok{"Fitted Values"}\NormalTok{, }\AttributeTok{y =} \StringTok{"Residuals"}\NormalTok{, }\AttributeTok{title =} \StringTok{"Residual vs. Fitted Plot (log transform)"}\NormalTok{)}
\end{Highlighting}
\end{Shaded}

\includegraphics{Lab2_files/figure-latex/unnamed-chunk-13-5.pdf}

\begin{Shaded}
\begin{Highlighting}[]
\CommentTok{\# scale location plot}
\FunctionTok{ggplot}\NormalTok{(logFull, }\FunctionTok{aes}\NormalTok{(.fitted, }\FunctionTok{sqrt}\NormalTok{(}\FunctionTok{abs}\NormalTok{(.stdresid)))) }\SpecialCharTok{+}
  \FunctionTok{geom\_point}\NormalTok{() }\SpecialCharTok{+}
  \FunctionTok{geom\_smooth}\NormalTok{(}\AttributeTok{se =} \ConstantTok{FALSE}\NormalTok{) }\SpecialCharTok{+}
  \FunctionTok{labs}\NormalTok{(}\AttributeTok{x =} \StringTok{"Fitted Values"}\NormalTok{, }\AttributeTok{y =} \StringTok{"Sqrt(|Standardized Residuals|)"}\NormalTok{, }\AttributeTok{title =} \StringTok{"Scale{-}Location Plot (log transform)"}\NormalTok{)}
\end{Highlighting}
\end{Shaded}

\begin{verbatim}
## `geom_smooth()` using method = 'loess' and formula = 'y ~ x'
\end{verbatim}

\includegraphics{Lab2_files/figure-latex/unnamed-chunk-13-6.pdf}

\begin{Shaded}
\begin{Highlighting}[]
\FunctionTok{avPlots}\NormalTok{(full\_model)}
\end{Highlighting}
\end{Shaded}

\includegraphics{Lab2_files/figure-latex/unnamed-chunk-13-7.pdf}

\textbf{Which model did you choose? Why? Are you worried about any
assumptions}\\
Based on my own (hopefully sound) intuition, as well as the results from
the VIF tests, I have decided to go with the full model including all
covariates.\\
I believe it intuitively makes sense that age and weight would play a
role in how much toluene is in the bloodstream, and snout size could be
important as to how much actually gets absorbed into the body.\\
I also believe it makes intuitive sense that none of these variables
would have much of heavy influence on each other or be strongly
correlated.\\
It is concerning that the residuals have a pattern in the plot
corresponding with the newppm variable, as this is the most important
factor in blood toluene levels. While similar to the residuals qq-plot
in the model just including newppm, the plot using the full model
appears very slightly more normal than the former. All of this could
indicate a potential non-linear relationship requiring transformation. A
log transform on blood toluene indeed shows more normal residuals, but
it also shows a much less horizontal line in the scale-location plot.

\section{Part 6) Identifying Influential or Outlier
Points}\label{part-6-identifying-influential-or-outlier-points}

\textbf{Based on the model you chose in part 5, identify poorly fit,
high leverage points, and/or influential points in your model.}
\textbf{Comment on how you identified these particular points (based on
which metric(s)?).}

\begin{Shaded}
\begin{Highlighting}[]
\CommentTok{\# the influence index plotting functions from the \textasciigrave{}car\textasciigrave{} (such as influenceIndexPlot()) may be useful to spot these points}
\FunctionTok{influenceIndexPlot}\NormalTok{(full\_model)}
\end{Highlighting}
\end{Shaded}

\includegraphics{Lab2_files/figure-latex/unnamed-chunk-14-1.pdf}

Points identified and why: Even originally from just look at the
added-variable (AV) plots, points 33 and 55 specifically appear to be
influential outliers. This is also seen in the Cook's Distance plot. On
the weight AV plot, points \textbf{27 and 11} appear to be particularly
high leverage, which is verified by the hat values plot. Based on all of
this, there is evidence to suggest that points \textbf{33 and 55},
specifically, are very anomalous, possibly due to measurement error.

\section{Part 7) Sensitivity to Outliers/Influential
Points}\label{part-7-sensitivity-to-outliersinfluential-points}

\textbf{Perform a sensitivity analysis by refitting the model excluding
the troublesome points in part 6. Is your model sensitive to these
points?}

\begin{Shaded}
\begin{Highlighting}[]
\CommentTok{\# fit model removing outlier/influential points}
\CommentTok{\# you can create a new data frame with the points remove or input the filtering in the lm() function}
\CommentTok{\# for examplem if I wanted to remove the second and tenth rows I could run: lm(y \textasciitilde{} x1 + x2, data = mydata \%\textgreater{}\% slice({-}c(2,10)))}
\CommentTok{\# the slice() function grabs or removes (if included with a \textasciigrave{}{-}\textasciigrave{} sign) rows by their index}
\NormalTok{dataClean }\OtherTok{\textless{}{-}}\NormalTok{ data }\SpecialCharTok{\%\textgreater{}\%}
  \FunctionTok{slice}\NormalTok{(}\SpecialCharTok{{-}}\FunctionTok{c}\NormalTok{(}\DecValTok{11}\NormalTok{, }\DecValTok{27}\NormalTok{, }\DecValTok{33}\NormalTok{, }\DecValTok{55}\NormalTok{))}

\NormalTok{modelClean }\OtherTok{\textless{}{-}} \FunctionTok{lm}\NormalTok{(bloodtol }\SpecialCharTok{\textasciitilde{}}\NormalTok{ newppm }\SpecialCharTok{+}\NormalTok{ weight }\SpecialCharTok{+}\NormalTok{ age }\SpecialCharTok{+}\NormalTok{ snout\_f, }\AttributeTok{data =}\NormalTok{ dataClean)}

\CommentTok{\# showing results}
\FunctionTok{summary}\NormalTok{(modelClean)}
\end{Highlighting}
\end{Shaded}

\begin{verbatim}
## 
## Call:
## lm(formula = bloodtol ~ newppm + weight + age + snout_f, data = dataClean)
## 
## Residuals:
##     Min      1Q  Median      3Q     Max 
## -9.8679 -2.1318 -0.6386  1.6518 11.7309 
## 
## Coefficients:
##               Estimate Std. Error t value Pr(>|t|)    
## (Intercept) -13.941114  10.530972  -1.324   0.1915    
## newppm        0.038263   0.002153  17.770   <2e-16 ***
## weight       -0.010411   0.013285  -0.784   0.4369    
## age           0.203607   0.107183   1.900   0.0631 .  
## snout_flong   0.969478   1.829679   0.530   0.5985    
## ---
## Signif. codes:  0 '***' 0.001 '**' 0.01 '*' 0.05 '.' 0.1 ' ' 1
## 
## Residual standard error: 4.341 on 51 degrees of freedom
## Multiple R-squared:  0.8892, Adjusted R-squared:  0.8805 
## F-statistic: 102.3 on 4 and 51 DF,  p-value: < 2.2e-16
\end{verbatim}

\begin{Shaded}
\begin{Highlighting}[]
\FunctionTok{avPlots}\NormalTok{(modelClean)}
\end{Highlighting}
\end{Shaded}

\includegraphics{Lab2_files/figure-latex/unnamed-chunk-15-1.pdf}

\begin{Shaded}
\begin{Highlighting}[]
\FunctionTok{influencePlot}\NormalTok{(full\_model)}
\end{Highlighting}
\end{Shaded}

\includegraphics{Lab2_files/figure-latex/unnamed-chunk-15-2.pdf}

\begin{verbatim}
##        StudRes        Hat        CookD
## 11 -0.28769762 0.46325756 0.0145299087
## 33 -4.24416254 0.14259609 0.4576025856
## 37  0.05673578 0.28647789 0.0002632515
## 55  4.42447248 0.07937806 0.2523474585
\end{verbatim}

\begin{Shaded}
\begin{Highlighting}[]
\FunctionTok{influencePlot}\NormalTok{(modelClean)}
\end{Highlighting}
\end{Shaded}

\includegraphics{Lab2_files/figure-latex/unnamed-chunk-15-3.pdf}

\begin{verbatim}
##       StudRes        Hat       CookD
## 34 -0.4261957 0.46314713 0.031852032
## 44  2.5843492 0.09635274 0.128158254
## 48  3.1716091 0.14493492 0.289570632
## 54 -2.5746760 0.13431952 0.185262828
## 56 -0.2372247 0.39221712 0.007400132
\end{verbatim}

\begin{Shaded}
\begin{Highlighting}[]
\CommentTok{\# comparing the two models}
\NormalTok{glance\_full }\OtherTok{\textless{}{-}}\NormalTok{ broom}\SpecialCharTok{::}\FunctionTok{glance}\NormalTok{(full\_model)}
\NormalTok{glance\_filtered }\OtherTok{\textless{}{-}}\NormalTok{ broom}\SpecialCharTok{::}\FunctionTok{glance}\NormalTok{(modelClean)}

\CommentTok{\# Comparing coefficients of the models}
\NormalTok{tidy\_full }\OtherTok{\textless{}{-}}\NormalTok{ broom}\SpecialCharTok{::}\FunctionTok{tidy}\NormalTok{(full\_model)}
\NormalTok{tidy\_filtered }\OtherTok{\textless{}{-}}\NormalTok{ broom}\SpecialCharTok{::}\FunctionTok{tidy}\NormalTok{(modelClean)}

\NormalTok{glance\_comparison }\OtherTok{\textless{}{-}} \FunctionTok{bind\_rows}\NormalTok{(glance\_full, glance\_filtered)}
\NormalTok{tidy\_comparison }\OtherTok{\textless{}{-}} \FunctionTok{bind\_rows}\NormalTok{(tidy\_full, tidy\_filtered)}

\NormalTok{knitr}\SpecialCharTok{::}\FunctionTok{kable}\NormalTok{(glance\_comparison)}
\end{Highlighting}
\end{Shaded}

\begin{longtable}[]{@{}
  >{\raggedleft\arraybackslash}p{(\columnwidth - 22\tabcolsep) * \real{0.0917}}
  >{\raggedleft\arraybackslash}p{(\columnwidth - 22\tabcolsep) * \real{0.1284}}
  >{\raggedleft\arraybackslash}p{(\columnwidth - 22\tabcolsep) * \real{0.0826}}
  >{\raggedleft\arraybackslash}p{(\columnwidth - 22\tabcolsep) * \real{0.0917}}
  >{\raggedleft\arraybackslash}p{(\columnwidth - 22\tabcolsep) * \real{0.0734}}
  >{\raggedleft\arraybackslash}p{(\columnwidth - 22\tabcolsep) * \real{0.0275}}
  >{\raggedleft\arraybackslash}p{(\columnwidth - 22\tabcolsep) * \real{0.0917}}
  >{\raggedleft\arraybackslash}p{(\columnwidth - 22\tabcolsep) * \real{0.0826}}
  >{\raggedleft\arraybackslash}p{(\columnwidth - 22\tabcolsep) * \real{0.0826}}
  >{\raggedleft\arraybackslash}p{(\columnwidth - 22\tabcolsep) * \real{0.0917}}
  >{\raggedleft\arraybackslash}p{(\columnwidth - 22\tabcolsep) * \real{0.1101}}
  >{\raggedleft\arraybackslash}p{(\columnwidth - 22\tabcolsep) * \real{0.0459}}@{}}
\toprule\noalign{}
\begin{minipage}[b]{\linewidth}\raggedleft
r.squared
\end{minipage} & \begin{minipage}[b]{\linewidth}\raggedleft
adj.r.squared
\end{minipage} & \begin{minipage}[b]{\linewidth}\raggedleft
sigma
\end{minipage} & \begin{minipage}[b]{\linewidth}\raggedleft
statistic
\end{minipage} & \begin{minipage}[b]{\linewidth}\raggedleft
p.value
\end{minipage} & \begin{minipage}[b]{\linewidth}\raggedleft
df
\end{minipage} & \begin{minipage}[b]{\linewidth}\raggedleft
logLik
\end{minipage} & \begin{minipage}[b]{\linewidth}\raggedleft
AIC
\end{minipage} & \begin{minipage}[b]{\linewidth}\raggedleft
BIC
\end{minipage} & \begin{minipage}[b]{\linewidth}\raggedleft
deviance
\end{minipage} & \begin{minipage}[b]{\linewidth}\raggedleft
df.residual
\end{minipage} & \begin{minipage}[b]{\linewidth}\raggedleft
nobs
\end{minipage} \\
\midrule\noalign{}
\endhead
\bottomrule\noalign{}
\endlastfoot
0.7748646 & 0.7584911 & 6.201280 & 47.32436 & 0 & 4 & -192.0113 &
396.0226 & 408.5887 & 2115.0731 & 55 & 60 \\
0.8891635 & 0.8804704 & 4.340679 & 102.28428 & 0 & 4 & -159.0516 &
330.1031 & 342.2552 & 960.9161 & 51 & 56 \\
\end{longtable}

\begin{Shaded}
\begin{Highlighting}[]
\NormalTok{knitr}\SpecialCharTok{::}\FunctionTok{kable}\NormalTok{(tidy\_comparison)}
\end{Highlighting}
\end{Shaded}

\begin{longtable}[]{@{}lrrrr@{}}
\toprule\noalign{}
term & estimate & std.error & statistic & p.value \\
\midrule\noalign{}
\endhead
\bottomrule\noalign{}
\endlastfoot
(Intercept) & -25.1906374 & 14.2830945 & -1.7636680 & 0.0833419 \\
newppm & 0.0331143 & 0.0027318 & 12.1215597 & 0.0000000 \\
weight & 0.0078489 & 0.0139483 & 0.5627148 & 0.5759157 \\
age & 0.3125911 & 0.1508750 & 2.0718558 & 0.0429812 \\
snout\_flong & -3.3974975 & 2.2743753 & -1.4938157 & 0.1409390 \\
(Intercept) & -13.9411137 & 10.5309720 & -1.3238202 & 0.1914661 \\
newppm & 0.0382632 & 0.0021533 & 17.7696631 & 0.0000000 \\
weight & -0.0104113 & 0.0132854 & -0.7836684 & 0.4368592 \\
age & 0.2036073 & 0.1071825 & 1.8996316 & 0.0631434 \\
snout\_flong & 0.9694779 & 1.8296794 & 0.5298622 & 0.5985068 \\
\end{longtable}

The new model with the troublesome points removed has a higher adjusted
R2 \textbf{(0.88 vs 0.76)} and very different coefficients for weight
and snout size, to the point that they change direction. Sigma decreased
in the filteread model as well. All this indicates that the model was
indeed very sensitive to these points.

\section{Part 8) Robust Regression}\label{part-8-robust-regression}

\textbf{Fit a robust regression using the model you chose in part 5 and
compare the two models. Comment on the differences. Do you think the
robust model is more appropriate here?}

\begin{Shaded}
\begin{Highlighting}[]
\CommentTok{\# fit robust regression model (rlm() funciton from \textasciigrave{}MASS\textasciigrave{} package)}
\NormalTok{modelRobust }\OtherTok{\textless{}{-}} \FunctionTok{rlm}\NormalTok{(bloodtol }\SpecialCharTok{\textasciitilde{}}\NormalTok{ newppm }\SpecialCharTok{+}\NormalTok{ weight }\SpecialCharTok{+}\NormalTok{ age }\SpecialCharTok{+}\NormalTok{ snout\_f, }\AttributeTok{data =}\NormalTok{ data)}

\CommentTok{\# showing summaries of the two models again}
\FunctionTok{summary}\NormalTok{(modelRobust)}
\end{Highlighting}
\end{Shaded}

\begin{verbatim}
## 
## Call: rlm(formula = bloodtol ~ newppm + weight + age + snout_f, data = data)
## Residuals:
##      Min       1Q   Median       3Q      Max 
## -23.4522  -1.7244  -0.4215   1.8844  25.5379 
## 
## Coefficients:
##             Value    Std. Error t value 
## (Intercept) -12.3493   8.1736    -1.5109
## newppm        0.0358   0.0016    22.8762
## weight       -0.0006   0.0080    -0.0775
## age           0.1561   0.0863     1.8079
## snout_flong  -0.6022   1.3015    -0.4627
## 
## Residual standard error: 2.578 on 55 degrees of freedom
\end{verbatim}

\begin{Shaded}
\begin{Highlighting}[]
\FunctionTok{summary}\NormalTok{(full\_model)}
\end{Highlighting}
\end{Shaded}

\begin{verbatim}
## 
## Call:
## lm(formula = bloodtol ~ newppm + weight + age + snout_f, data = data)
## 
## Residuals:
##      Min       1Q   Median       3Q      Max 
## -21.2982  -2.5936  -0.6433   2.0629  22.7613 
## 
## Coefficients:
##               Estimate Std. Error t value Pr(>|t|)    
## (Intercept) -25.190637  14.283095  -1.764   0.0833 .  
## newppm        0.033114   0.002732  12.122   <2e-16 ***
## weight        0.007849   0.013948   0.563   0.5759    
## age           0.312591   0.150875   2.072   0.0430 *  
## snout_flong  -3.397497   2.274375  -1.494   0.1409    
## ---
## Signif. codes:  0 '***' 0.001 '**' 0.01 '*' 0.05 '.' 0.1 ' ' 1
## 
## Residual standard error: 6.201 on 55 degrees of freedom
## Multiple R-squared:  0.7749, Adjusted R-squared:  0.7585 
## F-statistic: 47.32 on 4 and 55 DF,  p-value: < 2.2e-16
\end{verbatim}

\begin{Shaded}
\begin{Highlighting}[]
\NormalTok{coeff\_full }\OtherTok{\textless{}{-}} \FunctionTok{summary}\NormalTok{(full\_model)}\SpecialCharTok{$}\NormalTok{coefficients}
\NormalTok{coeff\_robust }\OtherTok{\textless{}{-}} \FunctionTok{summary}\NormalTok{(modelRobust)}\SpecialCharTok{$}\NormalTok{coefficients}

\CommentTok{\# making data.frame comparing the results}
\NormalTok{comparison\_df }\OtherTok{\textless{}{-}} \FunctionTok{data.frame}\NormalTok{(}
  \AttributeTok{Term =} \FunctionTok{rownames}\NormalTok{(coeff\_full),}
  \AttributeTok{Full\_Model\_Estimate =}\NormalTok{ coeff\_full[, }\StringTok{"Estimate"}\NormalTok{],}
  \AttributeTok{Robust\_Model\_Estimate =}\NormalTok{ coeff\_robust[, }\StringTok{"Value"}\NormalTok{],}
  \AttributeTok{Difference =}\NormalTok{ coeff\_full[, }\StringTok{"Estimate"}\NormalTok{] }\SpecialCharTok{{-}}\NormalTok{ coeff\_robust[, }\StringTok{"Value"}\NormalTok{]}
\NormalTok{)}

\NormalTok{knitr}\SpecialCharTok{::}\FunctionTok{kable}\NormalTok{(comparison\_df)}
\end{Highlighting}
\end{Shaded}

\begin{longtable}[]{@{}
  >{\raggedright\arraybackslash}p{(\columnwidth - 8\tabcolsep) * \real{0.1538}}
  >{\raggedright\arraybackslash}p{(\columnwidth - 8\tabcolsep) * \real{0.1538}}
  >{\raggedleft\arraybackslash}p{(\columnwidth - 8\tabcolsep) * \real{0.2564}}
  >{\raggedleft\arraybackslash}p{(\columnwidth - 8\tabcolsep) * \real{0.2821}}
  >{\raggedleft\arraybackslash}p{(\columnwidth - 8\tabcolsep) * \real{0.1538}}@{}}
\toprule\noalign{}
\begin{minipage}[b]{\linewidth}\raggedright
\end{minipage} & \begin{minipage}[b]{\linewidth}\raggedright
Term
\end{minipage} & \begin{minipage}[b]{\linewidth}\raggedleft
Full\_Model\_Estimate
\end{minipage} & \begin{minipage}[b]{\linewidth}\raggedleft
Robust\_Model\_Estimate
\end{minipage} & \begin{minipage}[b]{\linewidth}\raggedleft
Difference
\end{minipage} \\
\midrule\noalign{}
\endhead
\bottomrule\noalign{}
\endlastfoot
(Intercept) & (Intercept) & -25.1906374 & -12.3492994 & -12.8413380 \\
newppm & newppm & 0.0331143 & 0.0357627 & -0.0026485 \\
weight & weight & 0.0078489 & -0.0006188 & 0.0084677 \\
age & age & 0.3125911 & 0.1560909 & 0.1565003 \\
snout\_flong & snout\_flong & -3.3974975 & -0.6021871 & -2.7953104 \\
\end{longtable}

From this comparison, we can see that:

The intercept has a large difference, which indicates that the baseline
level predicted by the two models varies considerably.\\
The coefficient for \textbf{newppm is slightly higher in the robust
model}, but the difference is relatively small.\\
The \textbf{weight coefficient changes sign} between the two models,
which could indicate that this variable's influence is quite sensitive
to outliers or leverage points.\\
The \textbf{age coefficient is about half} in the robust model compared
to the full model, suggesting that the effect of age is less pronounced
when the influence of outliers is mitigated.\\
The \textbf{snout\_flong coefficient has the most significant change in
magnitude}, showing a drastic reduction in the robust model, which
implies that the full model's estimate for this term may be highly
influenced by outliers.

Based on all of this, and considering our previous statement of the
outliers being likely anomalies due to e.g.~measurement error, the
robust model may be more appropriate here as it reduces the impact of
these outliers.

\section{Part 9) Data
Transformations}\label{part-9-data-transformations}

\textbf{Based on what you saw in your descriptives from part 1 and the
residual plots from your final model would a data transformation be
appropriate? If so which one? (no need to transform, just comment).}

I believe a data transformation would be appropriate. As was seen
earlier, the residuals do not seem to be randomly scattered around the
horizontal line in the residual vs fitted plot and, instead, show some
pattern(s). This could help more clearly satisfy the assumptions of
linear regression. Additionally, using transformations could mitigate
the effect of some of the outliers, making the non-robust model more
appropriate to use.

\section{Part 10) Summary}\label{part-10-summary}

\textbf{Summarize your findings from parts 1-9} \textbf{(model chosen,
trends found using said model, possible multicollinearity, sensitivity
to outliers/influential points, etc.)}

\begin{Shaded}
\begin{Highlighting}[]
\FunctionTok{sessionInfo}\NormalTok{()}
\end{Highlighting}
\end{Shaded}

\begin{verbatim}
## R version 4.3.2 (2023-10-31 ucrt)
## Platform: x86_64-w64-mingw32/x64 (64-bit)
## Running under: Windows 11 x64 (build 25314)
## 
## Matrix products: default
## 
## 
## locale:
## [1] LC_COLLATE=English_United States.1252  LC_CTYPE=English_United States.1252    LC_MONETARY=English_United States.1252 LC_NUMERIC=C                          
## [5] LC_TIME=English_United States.1252    
## 
## time zone: America/New_York
## tzcode source: internal
## 
## attached base packages:
## [1] stats     graphics  grDevices utils     datasets  methods   base     
## 
## other attached packages:
##  [1] broom_1.0.5      corrplot_0.92    skimr_2.1.5      here_1.0.1       rlang_1.1.3      jtools_2.2.2     kableExtra_1.4.0 ggh4x_0.2.8      gridExtra_2.3    ggthemr_1.1.0    plotly_4.10.4   
## [12] MASS_7.3-60.0.1  car_3.1-2        carData_3.0-5    psych_2.4.1      lubridate_1.9.3  forcats_1.0.0    stringr_1.5.1    dplyr_1.1.4      purrr_1.0.2      readr_2.1.5      tidyr_1.3.1     
## [23] tibble_3.2.1     ggplot2_3.4.4    tidyverse_2.0.0  knitr_1.45      
## 
## loaded via a namespace (and not attached):
##  [1] tidyselect_1.2.0   viridisLite_0.4.2  farver_2.1.1       fastmap_1.1.1      lazyeval_0.2.2     digest_0.6.34      timechange_0.3.0   lifecycle_1.0.4    magrittr_2.0.3     compiler_4.3.2    
## [11] tools_4.3.2        yaml_2.3.8         utf8_1.2.4         data.table_1.14.10 labeling_0.4.3     htmlwidgets_1.6.4  bit_4.0.5          mnormt_2.1.1       xml2_1.3.6         repr_1.1.6        
## [21] abind_1.4-5        withr_3.0.0        grid_4.3.2         httpgd_1.3.1       fansi_1.0.6        colorspace_2.1-0   scales_1.3.0       cli_3.6.2          rmarkdown_2.25     crayon_1.5.2      
## [31] generics_0.1.3     rstudioapi_0.15.0  httr_1.4.7         tzdb_0.4.0         pander_0.6.5       splines_4.3.2      parallel_4.3.2     base64enc_0.1-3    vctrs_0.6.5        Matrix_1.6-5      
## [41] jsonlite_1.8.8     hms_1.1.3          bit64_4.0.5        systemfonts_1.0.5  glue_1.7.0         stringi_1.8.3      gtable_0.3.4       later_1.3.2        munsell_0.5.0      pillar_1.9.0      
## [51] htmltools_0.5.7    R6_2.5.1           rprojroot_2.0.4    vroom_1.6.5        evaluate_0.23      lattice_0.22-5     highr_0.10         backports_1.4.1    Rcpp_1.0.12        svglite_2.1.3     
## [61] nlme_3.1-164       mgcv_1.9-1         xfun_0.41          pkgconfig_2.0.3
\end{verbatim}

\end{document}
